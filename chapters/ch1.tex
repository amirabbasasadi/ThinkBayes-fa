\chapter{قضیه‌ی بیز}
\section{احتمال شرطی}
ایده‌ی بنیادی پشت تمام آمار بیزی قضیه‌ی بیز\LTRfootnote{Bayes's Theorem} است، قضیه ای که به طرز شگفت انگیزی بدست آوردنش آسان بوده و همچنین امکان درک احتمال شرطی را فراهم می کند.بحث را از احتمال آغاز کرده، سپس احتمال شرطی، قضیه‌‌ی بیز و در آخر آمار بیزی را بررسی می کنیم.\\\\
احتمال عددی بین $0$ و $1$ (شامل این دو هم می شود) و نشانگر درجه‌ی باور ما به یک حقیقت یا پیشبینی است. مقدار $1$ نشان دهنده اطمینان از درستی یک حقیقت و یا اطمینان از به وقوع پیوستن یک پیشبینی است. مقدار $0$ هم نشانگر نادرستی یک حقیقت است.\\\\
مقادیر میانی نشان دهنده‌ی درجه‌ی اطمینان و قطعیت هستند. مقدار $0.5$ که اغلب به صورت $50\%$ نوشته می شود، به این معناست که وقوع یک پیشبینی به اندازه‌ی عدم وقوع آن محتمل است. مثلا احتمال اینکه یک سکه‌ی پرتاب شده شیر باشد بسیار نزدیک $50\%$ است.\\\\
یک احتمال شرطی\LTRfootnote{Conditional Probability} احتمالی بر اساس برخی از اطلاعات قبلی است. برای مثال من می خواهم بدانم احتمال اینکه در سال آینده دچار یک حمله‌ی قلبی شوم چقدر است. بر طبق \lr{CDC}، سالانه حدود $785,000$ آمریکایی اولین حمله‌ی قلبی خود را تجربه می کنند.\LTRfootnote{\url{http://www.cdc.gov/heartdisease/facts.htm}}
جمعیت ایالات متحده حدودا $311$ میلیون نفر است، بنابراین احتمال اینکه یک آمریکایی که به طور تصادفی انتخاب شده در سال آینده دچار یک حمله‌ی قلبی شود، حدودا $0.3\%$ است. اما من که یک آمریکایی که به صورت تصادفی انتخاب شده باشد نیستم.همه گیر شناسان پی برده اند که عوامل زیادی بر میزان خطر وقوع حمله‌ی قلبی تاثیر می گذارند؛ بر اساس آن عوامل، احتمال اتفاق افتادن یک حمله‌ی قلبی برای من می تواند بالاتر یا پایین تر از میانگین باشد.\\
من یک مرد هستم، $45$ سال سن و کلسترول تقریبا بالایی دارم.این عوامل شانس ابتلاء را افزایش می دهند. در حالیکه، فشار خون من پایین است، سیگار نمی کشم و  این عوامل شانس را کاهش می دهند. من با وارد کردن اطلاعاتم در یک محاسبه کننده‌ی آنلاین به آدرس \url{http://cvdrisk.nhlbi.nih.gov/calculator.asp} متوجه شدم احتمال اینکه سال آینده یک حمله‌ی قلبی را تجربه کنم چیزی حدود $0.2\%$ و کمتر از میانگین است.این مقدار یک احتمال شرطی است چون بر اساس عواملی است که شرایط من محسوب می شوند.\\\\
معمولا احتمال شرطی را به صورت $p(A|B)$ نمایش می دهند که معنای آن احتمال درستی $A$ با فرض درستی $B$ است.در این مثال، $A$ نشان دهنده‌ی ابتلای من به حمله‌ی قلبی در سال آینده و $B$ نشان دهنده‌‌ی شرایطی که نام بردم است.
\section{احتمال توأم}
\section{مسئله‌ی کوکی}
\section{قضیه‌ی بیز}
\section{تفسیر در طول زمان}
\section{مسئله‌ی \lr{M \& M}}
\section{مسئله‌ی مونتی هال}
\section{بحث}

